%% Generated by Sphinx.
\def\sphinxdocclass{report}
\documentclass[letterpaper,10pt,english]{sphinxmanual}
\ifdefined\pdfpxdimen
   \let\sphinxpxdimen\pdfpxdimen\else\newdimen\sphinxpxdimen
\fi \sphinxpxdimen=.75bp\relax

\usepackage[utf8]{inputenc}
\ifdefined\DeclareUnicodeCharacter
 \ifdefined\DeclareUnicodeCharacterAsOptional
  \DeclareUnicodeCharacter{"00A0}{\nobreakspace}
  \DeclareUnicodeCharacter{"2500}{\sphinxunichar{2500}}
  \DeclareUnicodeCharacter{"2502}{\sphinxunichar{2502}}
  \DeclareUnicodeCharacter{"2514}{\sphinxunichar{2514}}
  \DeclareUnicodeCharacter{"251C}{\sphinxunichar{251C}}
  \DeclareUnicodeCharacter{"2572}{\textbackslash}
 \else
  \DeclareUnicodeCharacter{00A0}{\nobreakspace}
  \DeclareUnicodeCharacter{2500}{\sphinxunichar{2500}}
  \DeclareUnicodeCharacter{2502}{\sphinxunichar{2502}}
  \DeclareUnicodeCharacter{2514}{\sphinxunichar{2514}}
  \DeclareUnicodeCharacter{251C}{\sphinxunichar{251C}}
  \DeclareUnicodeCharacter{2572}{\textbackslash}
 \fi
\fi
\usepackage{cmap}
\usepackage[T1]{fontenc}
\usepackage{amsmath,amssymb,amstext}
\usepackage{babel}
\usepackage{times}
\usepackage[Bjarne]{fncychap}
\usepackage[dontkeepoldnames]{sphinx}

\usepackage{geometry}

% Include hyperref last.
\usepackage{hyperref}
% Fix anchor placement for figures with captions.
\usepackage{hypcap}% it must be loaded after hyperref.
% Set up styles of URL: it should be placed after hyperref.
\urlstyle{same}

\addto\captionsenglish{\renewcommand{\figurename}{Fig.}}
\addto\captionsenglish{\renewcommand{\tablename}{Table}}
\addto\captionsenglish{\renewcommand{\literalblockname}{Listing}}

\addto\captionsenglish{\renewcommand{\literalblockcontinuedname}{continued from previous page}}
\addto\captionsenglish{\renewcommand{\literalblockcontinuesname}{continues on next page}}

\addto\extrasenglish{\def\pageautorefname{page}}





\title{LoFASM Documentation}
\date{Apr 14, 2020}
\release{1}
\author{Louis Dartez}
\newcommand{\sphinxlogo}{\vbox{}}
\renewcommand{\releasename}{Release}
\makeindex

\begin{document}

\maketitle
\sphinxtableofcontents
\phantomsection\label{\detokenize{index::doc}}

\index{LofasmFile (class in lofasm.bbx.bbx)}

\begin{fulllineitems}
\phantomsection\label{\detokenize{index:lofasm.bbx.bbx.LofasmFile}}\pysiglinewithargsret{\sphinxbfcode{class }\sphinxcode{lofasm.bbx.bbx.}\sphinxbfcode{LofasmFile}}{\emph{lofasm\_file}, \emph{header=\{'metadata': \{\}\}}, \emph{verbose=False}, \emph{mode='read'}, \emph{gz=None}}{}
Class to represent .bbx-type data files for LoFASM.
Currently, the only data format supported is ‘LoFASM-filterbank’.
\index{add\_data() (lofasm.bbx.bbx.LofasmFile method)}

\begin{fulllineitems}
\phantomsection\label{\detokenize{index:lofasm.bbx.bbx.LofasmFile.add_data}}\pysiglinewithargsret{\sphinxbfcode{add\_data}}{\emph{data}}{}
add BBX data to memory to be written to file.

Write 1d or 2d data to memory to be written to disk.
If invoked with a new file then the dimension fields in the
metadata will be set.
\begin{quote}\begin{description}
\item[{Parameters}] \leavevmode
\sphinxstyleliteralstrong{data} (\sphinxstyleliteralemphasis{numpy.ndarray}) \textendash{} Data array to be added to memory
\sphinxtitleref{data.ndim} must be either 1 or 2.
The data type of the elements in the stored array will be inferred from \sphinxtitleref{data}.
Supported data types are np.complex128 and np.float64.

\item[{Raises}] \leavevmode\begin{itemize}
\item {} 
\sphinxcode{AssertionError} \textendash{} If file is not opened in write mode.

\item {} 
\sphinxcode{NotImplementedError} \textendash{} If the dimensions of \sphinxtitleref{data} are not supported.

\item {} 
\sphinxcode{ValueError} \textendash{} If either the number of channels in \sphinxtitleref{data} or the data type doesn’t match the pre-existing data, if there is
any.

\end{itemize}

\end{description}\end{quote}

\end{fulllineitems}

\index{close() (lofasm.bbx.bbx.LofasmFile method)}

\begin{fulllineitems}
\phantomsection\label{\detokenize{index:lofasm.bbx.bbx.LofasmFile.close}}\pysiglinewithargsret{\sphinxbfcode{close}}{}{}
close file object

in reading mode, the file handle is simply closed
in writing mode, write final header info to tmp file,
close tmp files, concat hdr and data portions, then
gzip the result if requested.

\end{fulllineitems}

\index{read\_data() (lofasm.bbx.bbx.LofasmFile method)}

\begin{fulllineitems}
\phantomsection\label{\detokenize{index:lofasm.bbx.bbx.LofasmFile.read_data}}\pysiglinewithargsret{\sphinxbfcode{read\_data}}{\emph{N=None}}{}
Parse data block in LoFASM filterbank file and load into
memory as \sphinxtitleref{self.data}.
The resulting data array in self.data is stored as a 2d
array with dim1 as the vertical axis and
dim2 as the horizontal axis.
If reading a typical LoFASM-filterbank file then the
x-axis will represent the time bins and the y-axis will
represent the frequency bins.
\begin{quote}\begin{description}
\item[{Parameters}] \leavevmode
\sphinxstyleliteralstrong{N} (\sphinxstyleliteralemphasis{int}) \textendash{} The number of rows to read. If not provided, then
attempt to read the entire file (read all rows).
If \sphinxtitleref{num\_time\_bin} is larger than the number of time
bins in the file then read the entire file.
A value of 0 will result in nothing being read.

\item[{Raises}] \leavevmode
\sphinxcode{AssertionError} \textendash{} If file is not open for reading

\end{description}\end{quote}

\end{fulllineitems}

\index{set() (lofasm.bbx.bbx.LofasmFile method)}

\begin{fulllineitems}
\phantomsection\label{\detokenize{index:lofasm.bbx.bbx.LofasmFile.set}}\pysiglinewithargsret{\sphinxbfcode{set}}{\emph{key}, \emph{val}}{}
Set header or metadata fields
Set or create header comment fields. If the field \sphinxtitleref{key} exists
then its value will be overwritten.
If the field does not exist, then it will be created as a
new comment field.
If \sphinxtitleref{key} exists as part of the metadata field, then the
value will be overwritten.
\begin{quote}\begin{description}
\item[{Parameters}] \leavevmode\begin{itemize}
\item {} 
\sphinxstyleliteralstrong{key} (\sphinxstyleliteralemphasis{str}) \textendash{} Header field name as a string.

\item {} 
\sphinxstyleliteralstrong{val} (\sphinxstyleliteralemphasis{str}\sphinxstyleliteralemphasis{, }\sphinxstyleliteralemphasis{int}\sphinxstyleliteralemphasis{, }\sphinxstyleliteralemphasis{float}) \textendash{} Value that header field will be set to

\end{itemize}

\end{description}\end{quote}

\end{fulllineitems}

\index{write() (lofasm.bbx.bbx.LofasmFile method)}

\begin{fulllineitems}
\phantomsection\label{\detokenize{index:lofasm.bbx.bbx.LofasmFile.write}}\pysiglinewithargsret{\sphinxbfcode{write}}{}{}
Write current data contents to disk.

If at the beginning of the file write the BBX header first, then the data.

\end{fulllineitems}


\end{fulllineitems}



\chapter{Indices and tables}
\label{\detokenize{index:indices-and-tables}}\label{\detokenize{index:welcome-to-lofasm-s-documentation}}\begin{itemize}
\item {} 
\DUrole{xref,std,std-ref}{genindex}

\end{itemize}



\renewcommand{\indexname}{Index}
\printindex
\end{document}